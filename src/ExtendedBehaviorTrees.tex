% Created 2014-12-03 Mi 07:52
\documentclass[a4wide]{article}
\usepackage[utf8]{inputenc}
\usepackage[T1]{fontenc}
\usepackage{fixltx2e}
\usepackage{graphicx}
\usepackage{longtable}
\usepackage{float}
\usepackage{wrapfig}
\usepackage{rotating}
\usepackage[normalem]{ulem}
\usepackage{amsmath}
\usepackage{textcomp}
\usepackage{marvosym}
\usepackage{wasysym}
\usepackage{amssymb}
\usepackage{hyperref}
\tolerance=1000
\usepackage{a4wide}
\parindent 0pt
\parskip 0.5ex
\author{Matthias Hölzl}
\date{\today}
\title{Extended Behavior Trees}
\hypersetup{
  pdfkeywords={},
  pdfsubject={},
  pdfcreator={Emacs 24.4.1 (Org mode 8.2.10)}}
\begin{document}

\maketitle
\tableofcontents


\section{About this document}
\label{sec-1}

The source of this document is an Org-Mode file that contains text,
Julia source code and some additional data such as spreadsheets or
tables with the results of experiments.  This file is completely
self-contained, so that it is straightforward to recreate the
results of the experiments once you have set up the required
environment.  To recompile the code in this document and re-run the
experiments you need to install Julia 0.4 as well as Emacs with
Org-Mode and ESS (Emacs Speaks Statistics).  If you lack parts of
this setup you will still be able to read the pre-generated output
(in \TeX{} and HTML format), and you can run the extracted Julia code
on data you provide to a stand-alone Julia implementation, but it
will be more difficult to recreate the results.

To evaluate all code in this file evaluate
\verb~org-babel-execute-buffer~ in Emacs (i.e., open the file in an
instance of Emacs in which the prerequisites mentioned in the
previous paragraph are installed, and enter \verb~M-x   org-babel-execute-buffer~).  Wait a while, and the results in the
buffer will be updated with the results of executing the examples on
your machine.

We add a warning (not shown in the typeset output) at the beginning
of the generated file \verb~ExtendedBehaviorTrees.jl~, so that people
don't accidentally edit the generated source.

The whole file is in the \verb~ExtendedBehaviorTrees~ module; the \verb~end~
for this module declaration is at the end of the file.  We disable
evaluation of the module statement inside Org-Mode since it causes
the Julia process to hang.  (Module disabled for now to simplify
reloading the code during development.)

\begin{verbatim}
module ExtendedBehaviorTrees
\end{verbatim}



\section{Behavior Trees and Extended Behavior Trees}
\label{sec-2}

Behavior Trees (BTs) are a flexible approach for behavioral
modeling.  BTs were originally introduced to model the AI of
computer games, and they have recently been become more popular in
areas such as robotics or avionics.  BTs compose atomic behaviors
(that should correspond to more or less simple actions an agent can
take) using operations such as sequence (execute multiple behaviors
one after the other) or choice (pick among behaviors until one
succeeds).  We sometimes call behaviors \emph{actions} or \emph{tasks}.  One
of the advantages of BTs is that the interface between a task and
its subtasks is very simple:

\begin{itemize}
\item Each task can be started; once started it runs until it returns
control to its parent.  Because of the root in game engines, this
behavior is often called \emph{ticking}, and the corresponding function
is \verb~tick~.  Scheduling is cooperative; behaviors have to return
after taking a small amount of time if the BT is integrated into a
frequently executing control loop.

\item A behavior can return one of three status indications: it
succeeded, it failed, or it is still running.  This last status
allows BTs to integrate long-running behaviors in an event loop.
\end{itemize}

\subsection{Extended Behavior Trees}
\label{sec-2-1}

\emph{Extended behavior trees (XBTs)} are an extension of behavior trees
that support various kinds of reasoning and learning beyond the
reactive planning of BTs.  To this end, XBTs have a slightly more
complicated interface between parent task and subtasks:

\begin{itemize}
\item The \verb~tick~ function is called with a \emph{state} parameter and still
returns three status indications.  However, when a task of an XBT
succeeds it passes back a measure of the quality of the solution
that it has achieved, and an indication whether it could continue
running to improve this solution.

\item All effects caused in the world have to be mediated by the state
passed in as argument.

\item The state can be \emph{virtualized}: Conceptually normal BTs operate
directly on the ``real world'' so that actions triggered by the
BT immediately affect the environment in which the agent running
the BT operates.  A virtual state decouples the actions of the BT
temporarily from the rest of the system, so that the agent can,
e.g., plan its future course of actions by observing the effects
various behaviors have on the world.
\end{itemize}

\subsection{Notes about the implementation}
\label{sec-2-2}

The implementation is not an exact clone of the SCEL semantics:

\begin{itemize}
\item There are two major strategies for implementing the evaluation of
the BT: 

\begin{itemize}
\item Regard the XBT instance as the description of the tree and
store all information for the evaluation in an external
environment.
\item Have a description layer above the classes that implement the
BT and store the data required for evaluating the tree directly
inside the nodes.
\end{itemize}
The first alternative is closer to the SCEL semantics, but the
second one seems slightly cleaner from a programming
perspective.

\item There is no reason why cloning the state should require a call to
an external service except that knowledge repositories are not a
first class concept in SCEL.  So here we simply expect the state
to implement a \verb~xbtclone~ method.
\end{itemize}

\section{Implementation of the basic function}
\label{sec-3}

In this section we set up the basics for XBTs: the results nodes may
return when ticking, the execution states of a node, and finally the
class structure of the node graph and the \verb~tick~ function.

\subsection{Execution status of nodes}
\label{sec-3-1}

Each node can be in one of four execution states: \emph{inactive},
\emph{running}, \emph{succeeded}, or \emph{failed}.  To avoid confusion with the
state passed to the \verb~tick~ function we cal this execution status of
a node its \emph{status}.  A node that has not yet been ticked has the
\emph{inactive} status.  Once it starts execution it transitions into the
\emph{running} status; when it returns a result or fails it moves into
either the \emph{succeeded} or \emph{failed} status.  In the \emph{succeeded}
status we keep track of the value the node achieved.

\begin{verbatim}
export XbtNodeStatus, Inactive, Running, Succeeded, Failed;

abstract XbtNodeStatus;
immutable Inactive <: XbtNodeStatus end;
immutable Running <: XbtNodeStatus end;
immutable Succeeded <: XbtNodeStatus
    value
end;
immutable Failed <: XbtNodeStatus end;
\end{verbatim}

We define predicates to test which status value we have.

\begin{verbatim}
export isinactive, isrunning, issucceeded, isfailed;

isinactive(x) = false;
isinactive(x::Inactive) = true;

isrunning(x) = false;
isrunning(x::Running) = true;

issucceeded(x) = false;
issucceeded(x::Succeeded) = true;

isfailed(x) = false;
isfailed(x::Failed) = true;
\end{verbatim}

\subsection{Results of ticking nodes}
\label{sec-3-2}

For each tick of an XBT, the nodes return either \emph{succeeded} (with
a quality value), \emph{failed} or \emph{running}, and an indication whether
they can improve the result they have obtained so far.  We
therefore return simply a tuple consisting of the status of the
node and a Boolean value.  The status in a result may never be an
instance of \verb~Inactive~; furthermore when the status is \emph{running}, the
second value has to be \verb~true~, when the status is \emph{failed}, the
second value has to be \verb~false~.

\begin{verbatim}
export XbtNodeResult;
typealias XbtNodeResult (XbtNodeStatus, Bool);
\end{verbatim}

Node results are typically used to determine whether we should
continue executing this node or not.  To simplify this we define a
function \verb~isdone~ that tells us whether we should continue after
obtaining a certain result.  \verb~isdone~ can either take an XBT node
(see definition below), an \verb~XbtNodeResult~, or a \verb~XbtNodeStatus~
and a Boolean value as arguments.

\begin{verbatim}
export isdone;
isdone(x) = false;
isdone(x::XbtNodeResult) = isdone(x...);
isdone(x, cont) = false;
isdone(x::Succeeded, cont::Bool) = !cont;
isdone(x::Failed, cont::Bool) = true;
\end{verbatim}

We define abbreviations for commonly used return values. When a
computation fails or wants to keep running we can simply return one
of the constants \verb~failed~ or \verb~running~; in these cases there is no
question whether the computation wants to continue or not, a failed
computation never wants to continue, a running computation always
wants to.  In the case of successful computation we have to return
a value, and either wanting to continue or not is possible.  Since
the former is the more likely case we make it the default.
\begin{verbatim}
export failed, running, succeeded;
const failed = (Failed(), false);
const running = (Running(), true);
succeeded(val, cont=false) = (Succeeded(val), cont);
\end{verbatim}

\subsection{XBT Nodes}
\label{sec-3-3}

Nodes in XBTs can either be composite (if they have children) or
atomic.  We might parameterize the classes on the type of the value
successful computations return, but this complicates the
definitions and does not seem to provide many benefits (since all
functions have type \verb~Function~, we cannot really use the type
parameter in the places where it might affect performance).

\begin{verbatim}
export XbtNode, AtomicXbtNode, CompositeXbtNode;
abstract XbtNode;
abstract AtomicXbtNode <: XbtNode;
abstract CompositeXbtNode <: XbtNode;
\end{verbatim}

Each task has to either store its execution status in a slot
\verb~status~ or provide a method on \verb~status~ so that we can determine
the execution status of tasks in a generic manner.  Similarly for
continuing with the node.

\begin{verbatim}
export status, setstatus;
status(node::XbtNode) = node.status;
setstatus(node::XbtNode, status::XbtNodeStatus) = node.status = status;

export cont, setcont;
cont(node::XbtNode) = node.cont;
setcont(node::XbtNode, cont::Bool) = node.cont = cont;

export result, setresult;
result(node::XbtNode) = status(node), cont(node);

function setresult(node::XbtNode, result::XbtNodeResult)
    setstatus(node, result[1])
    setcont(node, result[2])
    result
end;

isinactive(node::XbtNode) = isinactive(status(node));
isrunning(node::XbtNode) = isrunning(status(node));
issucceeded(node::XbtNode) = issucceeded(status(node));
isfailed(node::XbtNode) = isfailed(status(node));
isdone(node::XbtNode) = isdone(status(node), cont(node));

export setinactive;
function setinactive(node::XbtNode)
    setstatus(node, Inactive());
    setcont(node, true); # Slightly superfluous
end;
\end{verbatim}

We allow two kinds of atomic nodes: \verb~XbtTask~ and \verb~XbtFun~.  Tasks
are the more general nodes that are invoked as coroutines so that
they can suspend their computation while they are still running.
Instances of \verb~XbtFun~ are simply wrappers around functions that
succeed or fail but don't suspend.  (Maybe we should simply allow
functions as leaves?)

\begin{verbatim}
type XbtTask <: AtomicXbtNode
    task::Task
    status::XbtNodeStatus
    cont::Bool
end;

function XbtTask(fun::Function, status::XbtNodeStatus, cont::Bool)
    XbtTask(Task(fun), status, cont);
end;
XbtTask(task, status) = XbtTask(task, status, false);
XbtTask(task) = XbtTask(task, Inactive());

function tick(node::XbtTask, state)
    if (isdone(node))
        return result(node)
    end
    setresult(node, consume(node.task))
end;

type XbtFun <: AtomicXbtNode
    fun::Function
    status::XbtNodeStatus
    cont::Bool
end;

XbtFun(fun::Function, status::XbtNodeStatus) = XbtFun(task, status, false);
XbtFun(fun::Function) = XbtFun(fun, Inactive());

function tick(node::XbtFun, state)
    if (isdone(node))
        return result(node)
    end
    setresult(node, node.fun())
end;
\end{verbatim}

The following is a task that runs for three ticks, then succeeds
with value 1 and the possibility to continue.  When ticked after
succeeding for the first time it will continue to run for two more
ticks and then succeed with value 10 without being able to improve.
After that it will continue to succeed with value 10 until it is
reset to its initial state.


\begin{verbatim}
function f1()
    for i=1:3
        produce((Running(), true))
    end
    produce((Succeeded(1), true))
    produce((Running(), true))
    produce((Running(), true))
    produce((Succeeded(10), false))
end;
task1 = XbtTask(f1);
for i=1:8
    println(tick(task1, ()));
end;
\end{verbatim}

The results of this are example are as follows:
\begin{verbatim}
(Running(),true)
(Running(),true)
(Running(),true)
(Succeeded(1),true)
(Running(),true)
(Running(),true)
(Succeeded(10),false)
(Succeeded(10),false)
\end{verbatim}

The simplest kinds of composite XBT nodes are sequences and
choices.  A Sequence executes its child nodes sequentially until a
child node fails.  In that case the sequence node fails as well.
If all child nodes succeed the sequence node succeeds.  Choice
nodes work in the reverse manner: They execute their children in
turn until the first child succeeds in which case the choice
succeeds.  If all children fail the choice fails.  Since we will
later have several nodes that are sequence- or choice-like we
define abstract types for these two behaviors.

\begin{verbatim}
abstract XbtSequenceNode <: CompositeXbtNode;

function tick(node::XbtSequenceNode, state)
    local sum = 0, status, cont;
    for child in node.children
        status, cont = tick(child, state);
        if isa(status, Failed) return (status, false) end;
        if isa(status, Running) return (status, true) end;
        sum += status.value;
    end;
    # TODO: what about continuing?
    return Succeeded(sum, false), cont;
end;

immutable XbtSeq <: XbtSequenceNode
    children::AbstractArray{XbtNode,1}
end;

abstract XbtChoiceNode <: CompositeXbtNode;

function tick(node::XbtChoiceNode, state)
    local status, cont;
    for child in node.children
        status, cont = tick(child, state);
        if isa(status, Succeeded) return (status, cont) end;
        if isa(status, Running) return (status, true) end;
    end;
    return Failed(), cont;
end;

immutable XbtChoice <: XbtChoiceNode
    children::AbstractArray{XbtNode,1}
end;
\end{verbatim}

\section{HTN Planning}
\label{sec-4}

\subsection{States and Goals}
\label{sec-4-1}

\subsection{The Planner}
\label{sec-4-2}

\section{Reinforcement Learning}
\label{sec-5}

\begin{verbatim}
end; # module ExtendedBehaviorTrees
\end{verbatim}
% Emacs 24.4.1 (Org mode 8.2.10)
\end{document}